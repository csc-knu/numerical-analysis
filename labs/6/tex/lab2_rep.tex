\documentclass[12pt]{extarticle}
%Some packages I commonly use.
\usepackage[english, ukrainian]{babel}
\usepackage{graphicx}
\usepackage{framed}
\usepackage[normalem]{ulem}
\usepackage{amsmath}
\usepackage{amsthm}
\usepackage{amssymb}
\usepackage{amsfonts}
\usepackage{enumerate}
\usepackage{units}
\usepackage{xfrac}
\usepackage[utf8]{inputenc}
\usepackage[top=1 in,bottom=1in, left=1 in, right=1 in]{geometry}

%A bunch of definitions that make my life easier
\newcommand{\matlab}{{\sc Matlab} }
\newcommand{\cvec}[1]{{\mathbf #1}}
\newcommand{\rvec}[1]{\vec{\mathbf #1}}
\newcommand{\ihat}{\hat{\textbf{\i}}}
\newcommand{\jhat}{\hat{\textbf{\j}}}
\newcommand{\khat}{\hat{\textbf{k}}}
\newcommand{\minor}{{\rm minor}}
\newcommand{\trace}{{\rm trace}}
\newcommand{\spn}{{\rm Span}}
\newcommand{\rem}{{\rm rem}}
\newcommand{\ran}{{\rm range}}
\newcommand{\range}{{\rm range}}
\newcommand{\mdiv}{{\rm div}}
\newcommand{\proj}{{\rm proj}}
\newcommand{\R}{\mathbb{R}}
\newcommand{\N}{\mathbb{N}}
\newcommand{\Q}{\mathbb{Q}}
\newcommand{\Z}{\mathbb{Z}}
\newcommand{\<}{\langle}
\renewcommand{\>}{\rangle}
\renewcommand{\emptyset}{\varnothing}
\newcommand{\attn}[1]{\textbf{#1}}
\theoremstyle{definition}
\newtheorem{theorem}{Theorem}
\newtheorem{corollary}{Corollary}
\newtheorem*{definition}{Definition}
\newtheorem*{example}{Example}
\newtheorem*{note}{Note}
\newtheorem{exercise}{Exercise}
\newcommand{\bproof}{\bigskip {\bf Proof. }}
\newcommand{\eproof}{\hfill\qedsymbol}
\newcommand{\Disp}{\displaystyle}
\newcommand{\qe}{\hfill\(\bigtriangledown\)}
\setlength{\columnseprule}{1 pt}

\newcommand{\thickhat}[1]{\mathbf{\hat{\text{$#1$}}}}
\newcommand{\thickbar}[1]{\mathbf{\bar{\text{$#1$}}}}
\newcommand{\thicktilde}[1]{\mathbf{\tilde{\text{$#1$}}}}

\graphicspath{ {images/} }

\title{Звіт до лабораторної роботи №2 \\ Чисельне інтегрування}
\author{ОМ-3, Бабієнка Іллі}
\date{Березень 2019}

\begin{document}

\maketitle

\section{Постановка задачі}

Нехай задана функція $f(x)$. Необхідно наближено обчислити інтеграл $I=\int_{b}^{a}f(x)dx$ від $f(x)$ на заданому інтервалі $[a,b]$, де $a < b \wedge a,b \in \bar{\R}$.\\
Вивести отримане наближене значення інтегралу $I_h$, похибку $R_{h}^{\circ}$ та кінцевий крок розбиття інтервалу $h$.\\
(За замовчуванням використовуємо рівномірно-розподілені вузли)\\
Для обчислень використати формулу Сімпсона, принцип Рунге, формулу Річардсона, та апріорні оцінки похибки для квадратурних формул.
\begin{equation}
	f(x) = \frac{1}{x^2+x-2}, \space a = 2, \space b = +\infty
\end{equation}

\vspace{1cm}

\section{Теоретичні відомості}
\subsection{Формула Сімпсона}

Нехай $x_{0}=a, x_{1}=\frac{a+b}{2},x_{2}=b$. Замість \textit{f} викроистаємо $L_{2}(x)$. Тоді отримаємо квадратурну формулу:

\begin{equation}
	I_{2}(x) = \frac{b-a}{6}\left[f(a)+4f(\frac{a+b}{2})+f(b) \right]
\end{equation}

Це \textit{квадратурна формула Сімпсона}.
Для $f \in C^{4}[a,b]$ залишковий член квадратури має представлення:

\begin{equation}
	R_{2}(f)=\frac{1}{24}\int_{a}^{b}(x-a)(x-\frac{a+b}{2})^{2}(x-b)f^{(4)}(\xi)dx=\frac{f^{(4)}(\xi)}{2880}(b-a)^{5},
\end{equation}

та вірна оцінка ($M_{4} = \underset{x \in [a,b]}{\max}|f^{(4)}(x)|$):

\begin{equation}
	|R_{2}(f)|\leq \frac{M_{4}}{2880}(b-a)^{5}
\end{equation}

Складена формула Сімпсона (отримана застосуванням звичайної формули до кожного з інтервалів рівномірної сітки, на які поділений початковий інтервал):

\begin{equation}
	I_{h}(f) = \sum\limits_{i=1}^{N}\frac{h}{6}\left[f(x_{i-1}) + 4f(\frac{x_{i-1}+x_{i}}{2}) + f(x_{i})\right]
\end{equation}


Для неї справджуються наступні оцінки (апріорні):

\begin{itemize}

\item $f \in C^{4}[a,b]:$
\begin{equation}
|R_{h}(f)| \leq \frac{M_4}{2880}(b-a)h^4
\end{equation}

\item $f \in C^{6}[a,b]:$
\begin{equation}
 R_{h}(f)=R_{h}^{\circ}(f)+\alpha(h), \;  \text{де} \; R_{h}^{\circ}(f)=\frac{h^4}{2880}\int\limits_{a}^{b}f^{(4)}(x)dx, \; \alpha(h)=O(h^6) 
\end{equation}

\end{itemize}

\subsection{Принцип Рунге. Формула Річардсона}
\begin{center}
	% \makebox[width=0.8\textwidth]{\includegraphics[width=\paperwidth]{simpson1.png}}
	\includegraphics[width=\textwidth]{runge1.png}
\end{center}

\pagebreak

\begin{center}
	\includegraphics[width=\textwidth]{runge2.png}
	\includegraphics[width=\textwidth]{runge3.png}
	\includegraphics[width=\textwidth]{runge4.png}
\end{center}

Для формули Сімпсона (для якої $m=4$), формула Річардсона матиме вигляд:

\begin{equation}
	\thicktilde{I}_{h/2}=\frac{16}{15}I_{h/2}-\frac{1}{15}I_{h}, \quad I_h-\thicktilde{I}_{h/2}=O(h^6)
\end{equation}

\vspace{1cm}

\section{Практична частина}

Даний інтеграл є невласним інтегралом першого роду: $I = \int\limits_{2}^{\infty}\frac{dx}{x^2+x-2}$.
Перепишемо $f(x)$ як $\frac{1}{(x+2)(x-1)}$. Бачимо, що особливостей на області інтегрування немає (вони є в точках $x=-2$ та $x=1$, які лежать поза областю інтегрування). \\
Зробимо заміну: $t = \frac{x-2}{x},\ x=\frac{2}{1-t}, \ dx = \frac{2dt}{(1-t)^2}; \rightarrow \ I=\int\limits_{2}^{\infty}\frac{dx}{(x+2)(x-1)} = \int\limits_{0}^{1}\frac{dt}{(2-t)(1+t)} \quad (9)$ \\
Тоді інтеграл набуде вигляду (9), де $g(t) = \frac{1}{(2-t)(1+t)}$ - нова підінтегральна фунція, $[0,1]$ - новий проміжок інтегрування.
Отриманий інтеграл є власним інтегралом. Отже його можна обчислювати формулами Сімпсона та Річардсона напряму (без попередніх досліджень чи перетворень).

Також помітимо, що початковий (як і отриманий) інтеграл можна знайти аналітично: $f(x) = \frac{1}{(x+2)(x-1)} = \frac{1}{3}\left(\frac{1}{x-1} - \frac{1}{x+2}\right)$ - не складно інтегрується безпосереднім знаходженням первісної. Справжнє значення інтегралу рівне $\ln(4)/3 \approx 0.46209812037329687294..$.


Для обчислення апріорної похибки обчислень, нам необхідно оцінити четверту похідну фунції $g(t)$.
$g^{(4)}=\frac{24}{3}\left( \frac{1}{(2-t)^5} + \frac{1}{(1+t)^5}\right)$ \\
Вона є обмеженою та додатньою на всьому інтервалі інтегрування.\\
П'ята похідна $g^{(5)}=\frac{120}{3}\left( \frac{1}{(2-t)^6} - \frac{1}{(1+t)^6}\right)$
більша за 0 коли $t>1/2$ та менша за 0 коли $t<1/2$
Отже (локальні) максимуми доягаються на кінцях: $g^{(4)}(0)=g^{(4)}(1)=\nicefrac{33}{32}$.
Отже $M_4 = 33/32, $ та $|R_{2}(f)|\leq \frac{M_{4}}{2880}(b-a)^{5} = \frac{33}{32*2880} \approx 0.00035807291(6)$.\\
В методі Рунге виставимо точність $\epsilon = 10^{-3}$ та кількість ітерацій не більше 20. 

Результати:\\
% \begin{center}
\includegraphics[scale=0.8]{lab2_res.png}
% \end{center}


\end{document}
