\documentclass[12pt, a4paper]{article}
\usepackage[T2A]{fontenc}
\usepackage[utf8]{inputenc}
\usepackage[english,ukrainian]{babel}
\usepackage{amsmath, amssymb}
\usepackage{verbatim}

\usepackage[top = 2 cm, left = 1 cm, right = 1 cm, bottom = 2 cm]{geometry}

\usepackage{float, graphicx}
\usepackage{amsthm}
\newtheorem{lemma}{Лема}
\newtheorem*{lemma*}{Лема}
\newtheorem{theorem}{Теорема}
\newtheorem*{theorem*}{Теорема}
\newtheorem{definition}{Визначення}
\newtheorem*{definition*}{Визначення}
\theoremstyle{definition}
\newtheorem{remark}{Зауваження}
\newtheorem*{remark*}{Зауваження}
\newtheorem{example}{Приклад}
\newtheorem*{example*}{Приклад}
\newtheorem{problem}{Задача}
\newtheorem*{problem*}{Задача}
\newtheorem{solution}{Розв'язок}
\newtheorem*{solution*}{Розв'язок}
\newtheorem{corollary}{Наслідок}
\newtheorem*{corollary*}{Наслідок}

\newcommand{\NN}{\mathbb{N}}
\newcommand{\RR}{\mathbb{R}}
\newcommand{\CC}{\mathbb{C}}
\newcommand{\Min}{\displaystyle\min\limits}
\newcommand{\Max}{\displaystyle\max\limits}
\newcommand{\Sup}{\displaystyle\sup\limits}
\newcommand{\Sum}{\displaystyle\sum\limits}
\newcommand{\Prod}{\displaystyle\prod\limits}
\newcommand{\Int}{\displaystyle\int\limits}
\newcommand{\Iint}{\displaystyle\iint\limits}
\newcommand{\Lim}{\displaystyle\lim\limits}

\newcommand*\diff{\mathop{}\!\mathrm{d}}

\renewcommand{\bf}[1]{\textbf{#1}}
\renewcommand{\epsilon}{\varepsilon}
\renewcommand{\phi}{\varphi}

\DeclareMathOperator{\signum}{sign}
\DeclareMathOperator{\diam}{diam}
\DeclareMathOperator{\rang}{rang}
\DeclareMathOperator{\const}{const}
\DeclareMathOperator{\cond}{cond}

\numberwithin{equation}{section}

\setlength\parindent{0pt}
\allowdisplaybreaks

\newcommand{\cover}[2]{
\begin{center}
\hfill \break
	Міністерство освіти та науки України \\
	Київський національний університет імені Тараса Шевченка \\ 
	Факультет комп'ютерних наук та кібернетики \\
	Кафедра обчислювальної математики
\end{center}

\vfill 

\begin{center}
	\large{
		Звіт до лабораторної роботи №{#1} на тему: \\ 
		``{#2}'
	}
\end{center}

\vfill 

\begin{flushright}
	Виконав студент групи ОМ-3 \\
	Скибицький Нікіта
\end{flushright}

\vfill 

\begin{center}
    Київ, 2018 
\end{center}

\thispagestyle{empty} 
\newpage
}

\begin{document}

\cover{6}{Квадратурні формули}

\tableofcontents

\section{Постановка задачі}

Обчислити невласний інтеграл 

\begin{equation}
	I(f) = \Int_{-1}^1 \frac{\ln (2 + \sqrt[3]{x})}{\sqrt[3]{x}} \diff x.
\end{equation}

з точністю $\varepsilon = 10^{-5}$. \medskip

Для обчислень використати:

\begin{enumerate}
	\item формули:
	\begin{enumerate}
		\item середніх прямокутників;
		\item трапецій;
		\item Сімпсона (парабол).
	\end{enumerate}
	\item прицип Рунге;
	\item формулу Річардсона;
	\item апріорні оцінки похибки.
\end{enumerate}

Вивести:

\begin{enumerate}
	\item кінцевий крок інтегрування $h$;
	\item апріорну оцінку тчоності інтегрування;
	\item наближені значення інтегралу $I_h$, $I_{h/2}$;
	\item оцінку похибки інтегрування за принципом Рунге;
	\item уточнене значення інтегралу за формулою Річардсона.
\end{enumerate}

\section{Теоретичні відомості}

\subsection{Невласність}

Від невласності можна позбутися адитивним методом, тобто представивши підінтегральну функцію у вигляді суми функції з особливістю яку інтегруємо аналітично та не особливої функції яку інтегруємо чисельно:

\begin{align*}
	\Int_{a}^{b} f(x) \diff x &= \Int_{a}^{b} (x - x_0)^{-\alpha} \varphi(x) \diff x = \\
	&= \Int_{a}^{b} (x - x_0)^{-\alpha} \Sum_{k = 0}^\infty c_k (x - x_0)^k \diff x = \\
	&= \Int_{a}^{b} (x - x_0)^{-\alpha} \left( \Sum_{k = 0}^n c_k (x - x_0)^k + \Sum_{k = 1}^\infty c_{n + k} (x - x_0)^{n + k} \right) \diff x = \\
	&= \Int_{a}^{b} (x - x_0)^{-\alpha} \Sum_{k = 0}^n c_k (x - x_0)^k \diff x + \Int_{a}^{b} (x - x_0)^{-\alpha} \Sum_{k = 1}^\infty c_{n + k} (x - x_0)^{n + k} \diff x = \\
	&= \Int_{a}^{b} g(x) \diff x + \Int_{a}^{b} (x - x_0)^{-\alpha} \Sum_{k = 1}^\infty c_{n + k} (x - x_0)^{n + k} \diff x = \\
	&= \Int_{a}^{b} g(x) \diff x + \Int_{a}^{b} (x - x_0)^{-\alpha} \left( \Sum_{k = 0}^\infty c_{k} (x - x_0)^{k} - g(x) \right) \diff x = \\
	&= \Int_{a}^{b} g(x) \diff x + \Int_{a}^{b} (x - x_0)^{-\alpha} \left( \phi(x) - g(x) \right) \diff x = \\
	&= \Int_{a}^{b} g(x) \diff x + \Int_{a}^{b} \psi(x) \diff x,
\end{align*}

далі 

\begin{equation}
	\Int_{a}^{b} g(x) \diff x = \Int_{a}^{b} (x - x_0)^{-\alpha} \Sum_{k = 0}^n c_k (x - x_0)^k \diff x
\end{equation}

беремо аналітично а

\begin{equation}
	\Int_{a}^{b} \psi(x) \diff x = \Int_{a}^{b} \Sum_{k = 1}^{\infty} c_{n + k} (x - x_0)^{n + k - \alpha} \diff x
\end{equation}

--- чисельно. \medskip

Такий підхід у нашому випадку працює погано, адже для апріорної оцінки похибки чисельного інтегрування необхідна принаймні скінченна друга похідна у $\psi$, а для цього в $g$ доводиться брати багато (6) членів ряду Тейлора. А це у свою чергу призводить до малості $\psi$, що ускладнює точне обчислення її інтегралу за рахунок зростання відносної машинної похибки. \medskip

Тому ми просто зробимо заміну і зведемо інтеграл до власного.

\subsection{Квадратурні формули}

\subsubsection{Середніх прямокутників}

\begin{equation}
	I_0 = (b - a) \cdot f(x_0), \quad x_0 = \frac{a + b}{2}
\end{equation}

Знайдемо алгебраїчну степінь точності цієї квадратурниої формули:

\begin{align}
	I_0(1) &= (b - a) \cdot 1 = I(x), \\
	I_0(x) &= (b - a) \cdot \frac{a + b}{2} = I(x), \\
	I_0(x^2) &= (b - a) \cdot \left(\frac{a + b}{2}\right)^2 \ne \frac{b^3 - a^3}{3} = \Int_a^b x^2 \diff x = I(x^2),
\end{align}

тому $m = 1$. \medskip

Оцінимо для неї похибку. Взагалі для формули інтерполяційного типу:

\begin{equation}
	R_n(f) = I(f) - I_n(f) = I(f) - I(L_n) = I(f - L_n) = I(r_n) = \Int_a^b r_n(x) \diff x,
\end{equation}

де $r_n(x)$ --- залишковий член інтерполяції. Далі

\begin{equation}
	|R_n(f)| \le (b - a) \cdot \Max_x |r_n(x)| \le (b - a) \cdot \frac{M_{n + 1}}{n + 1} \cdot \Max_x |\omega_n(x)|.
\end{equation}

Для $I_0$:

\begin{equation}
	\begin{aligned}
		|R_0(f)| &= \left| \Int_a^n r_0(x) \diff x \right| \le \Int_a^b |r_0(x)| \diff x \le \Int_a^b \frac{M_1}{1!} |x - x_0| \diff x = \\
		&= M_1 \Int_a^b |x - x_0| \diff x \le M_1 \cdot \frac{b^2 - a^2}{4}.
	\end{aligned}
\end{equation}

Але це погана оцінка, вона не використовує той факт, що квадратурна формула має степінь точності на одиницю вищу. Отримаємо кращу оцінку. Маємо при $f \in C^2([a,b])$:

\begin{equation}
	f(x) = f(x_0) + (x - x_0) \cdot f'(x_0) + \frac{(x - x_0)^2}{2} \cdot f''(\xi),
\end{equation}

де $x_0 \equiv \frac{a + b}{2}$, а $\xi \in [a, b]$. Тоді

\begin{equation}
	\begin{aligned}
		R_0(f) &= \Int_a^b f(x) \diff x  \Int_a^b L_0(x) \diff x = \Int_a^b (f(x) - f(x_0)) \diff x = \\
		&= \Int_a^b \left((x - x_0) f'(x_0) + \frac{(x - x_0)^2}{2} f''(\xi)\right) \diff x = \\
		&= \Int_a^b \frac{(x - x_0)^2}{2} \cdot f''(\xi) \diff x = f''(\eta) \Int_a^b \frac{(x - x_0)^2}{2} \diff x = \frac{f''(\eta)}{24} \cdot (b - a)^3.
	\end{aligned}
\end{equation}

Таким чином

\begin{equation}
	|R_0(f)| \le \frac{M_2}{24} \cdot (b - a)^3
\end{equation}

Але тут у нас немає впливу на точність (величину похибки). Тому використовують формулу складеного типу. Якщо сітка рівномірна, то складена квадратурна формула прямокутників має вигляд

\begin{equation}
	I(f) = \Sum_{i = 1}^n \Int_{x_{i - 1}}^{x_i} f(x) \diff x \approx \Sum_{i = 1}^N h \cdot f ( \bar x_i) = I_h(f),
\end{equation}

де $\bar x_i = x_{i - 1/2} = x_i - h/2$. \medskip

Оцінимо похибку цієї квадратурної формули:

\begin{equation}
	R_h(f) = I(f) - I_h(f) = \Sum_{i = 1}^N \Int_{x_{i - 1}}^{x_i} (f(x) - f(\bar x_i)) \diff x = \Sum_{i = 1}^N f''(\eta_i) \cdot \frac{h^3}{24},
\end{equation}

\begin{equation}
	|R_h(f)| \le \frac{M_2}{24} n h^3 = \frac{M_2 h^2 (b - a)}{24}.
\end{equation}

Тобто ця формула має степінь точності $p = 2$ по кроку $h$. (Не слід плутати з алгебраїчним степенем точності $m = 1$ для цієї формули).

\subsubsection{Трапецій}

Нехай $x_0 = a$, $x_1 = b$, $L_1(x) = f(x)$. Тоді отримаємо формулу:

\begin{equation}
	I_1(f) = \frac{b - a}{2} \cdot (f(a) + f(b))
\end{equation}

Формула має алгебраїчний степінь точності $m = 1$, оскільки $I(x^2) \ne I_1(x^2)$. Це формула замкненого типу. Залишковий член:

\begin{equation}
	R_1(f) = \Int_a^b \frac{f''(\xi)(x - a)(x - b)}{2} \diff x = - \frac{(b - a)^3}{12} \cdot f''(\xi).
\end{equation}

Оцінка залишкового члена:

\begin{equation}
	|R_1(f)| \le M_2 \cdot \frac{(b - a)^3}{12}.
\end{equation}

З геометричної точки зору замінюється площа криволінійної трапецій площею звичайної трапеції. \medskip

Складена квадратурна формула трапецій:

\begin{equation}
	\begin{aligned}
		I_h(f) &= \Sum_{i = 1}^N \frac{h}{2} \cdot (f(x_{i - 1}) + f(x_i)) = \\
		&= \frac{h}{2} \cdot f(a) + \Sum_{i = 1}^{N - 1} h \cdot f(x_i) + \frac{h}{2} \cdot f(b),
	\end{aligned}
\end{equation}

де $x_i = a + i h$, $h = \frac{b - a}{N}$, $i = \overline{0, N}$ та

\begin{equation}
	|R_h(f)| \le M_2 \cdot \frac{b - a}{12} \cdot h^2, \quad f \in C^2([a,b]).
\end{equation}

\subsubsection{Сімпсона (парабол)}

Нехай $x_0 = a$, $x_1 = \frac{a + b}{2}$, $x_2 = b$. Замість $f$ використовуємо $L_2(x)$. Тоді отримаємо квадратурну формулу:

\begin{equation}
	I_2(f) = \frac{b - a}{6} \left( f(a) + 4 f \left( \frac{a + b}{2} \right) + f(b) \right).
\end{equation}

Це \textit{квадратурна формула Сімпсона}. \medskip

Алгебраїчна степінь точності квадратурної формули Сімпсона $m = 3$. \medskip

Для $f \in C^4([a,b])$ залишковий член квадратурної формули Сімпсона має місце представлення:

\begin{equation}
	R_2(f) = \frac{1}{24} \Int_a^b (x - a) \left( x - \frac{a + b}{2} \right)^2 (x - b) f^{(4)}(\xi) \diff x = \frac{f^{(4)}(\xi)}{2880} \cdot (b - a)^5,
\end{equation}

та вірна оцінка:

\begin{equation}
	|R_2(f)| \le \frac{M_4}{2880} \cdot (b - a)^5.
\end{equation}

Складена квадратурна формула Сімпсона має вигляд:

\begin{equation}
	\begin{aligned}
		I_h(f) &= \Sum_{i = 1}^N \frac{h}{6} \left( f(x_{i - 1}) + 4 f (x_{i - 1/2}) + f(x_i) \right) = \\
		&= \frac{h}{6} ( f(x_0) + 4 f(x_{1/2}) + 2 f(x_1) + \ldots + 2 f(x_{N - 1}) + 4 f(x_{N-1/2}) + f(x_N) ).
	\end{aligned}
\end{equation}

Якщо $f \in C^4([a,b])$, то має місце оцінка:

\begin{equation}
	|R_h(f)| \le \frac{M_4}{2880} \cdot (b - a) \cdot h^4, \quad p = 4.
\end{equation}

\subsection{Принцип Рунге і формула Річардсона}

Нехай задана деяка величина $I$ (сіткова функція, інтеграл, неперервна функція). Нехай $I_h \approx I$ та $I_n \to I$ при $h \to 0$. Нехай похибка послідовності $I_h$ представляється у вигляді:

\begin{equation}
	\label{eq:9.5.1}
	R_h = I - I_h = \overset{\circ} R_h + \alpha(h),
\end{equation}

де $\overset{\circ}{R}_h = C \cdot h^m$ --- головний член похибки, $C$ не залежить від $h$, $\alpha(h) = o(h^m)$. Обчислимо $I_{h/2}$. З \eqref{eq:9.5.1} випливає, що

\begin{align}
	I &= I_h + C h^m + \alpha(h), \\
	I &= I_{h/2} + C \cdot \frac{h^m}{2^m} + \alpha(h).
\end{align}

Звідси

\begin{equation}
	I_{h/2} - I_h = \frac{C h^m}{2^m} \cdot (2^m - 1) + \alpha(h).
\end{equation}

З \eqref{eq:9.5.1}:

\begin{equation}
	\label{eq:9.5.2}
	\overset{\circ} R_{h/2} = \frac{C h^m}{2^m} = \frac{I_{h/2} - I_h}{2^m - 1},
\end{equation}

та

\begin{equation}
	\overset{\circ} R_h = \frac{2^m}{2^m - 1} \cdot (I_{h/2} - I_h).
\end{equation}

Формула \eqref{eq:9.5.2} носить назву \textit{апостеріорної оцінки} похибки обчислення $I$ за допомогою наближення $I_{h/2}$. (\textit{Апріорні оцінки} це оцінки отримані до обчислення величини $I_h$, \textit{апостеріорні оцінки} --- під час її обчислення). \medskip

З формули \eqref{eq:9.5.2} впливає такий алгоритм обчислення інтегралу із заданою точністю $\varepsilon$:

\begin{enumerate}
	\item обчислюємо $I_h$, $I_{h/2}$, $\overset{\circ} R_{h/2}$; 

	\item перевіряємо чи $\left\vert \overset{\circ} R_{h/2} \right\vert < \varepsilon$.

	\item Якщо так, то $I \approx I_{h/2}$;

	\item Якщо ж ні, то:

	\begin{enumerate}
		\item обчислюємо $I_{h/2}$, $I_{h/4}$, $\overset{\circ} R_{h/4}$; 

		\item перевіряємо $\left\vert \overset{\circ} R_{h/4} \right\vert < \varepsilon$ і т. д.
	\end{enumerate}

	\item Процес продовжуємо поки не буде виконана умова $\left\vert \overset{\circ} R_{h/2^k} \right\vert < \varepsilon$, $k = 1,2, \ldots$.
\end{enumerate}


\textbf{Зауваження}: Ми даємо оцінку не похибки, а її головного члена з точністю $\alpha(h)$, тому такий метод може давати збої, якщо не виконана умова

\begin{equation}
	|\alpha(h)| \ll \left| \overset{\circ} R_{h/2^k} \right|.
\end{equation}

За допомогою головного члена похибки можна отримати краще значення для $I$:

\begin{equation}
	\label{eq:9.5.3}
	\tilde I_{h/2} = I_{h/2}^{(1)} = I_{h/2} + \overset{\circ} R_{h/2} = \frac{2^m}{2^m - 1} \cdot I_{h/2} - \frac{1}{2^m - 1} \cdot I_h.
\end{equation}

Це \textit{екстраполяційна формула Річардсона}: $I_h - \tilde I_{h/2} = \alpha(h)$. \medskip

Для квадратурної формули трапецій $p = 2$ і 

\begin{align}
	I - I_h &= C h^2 + O(h^4), \\
	\overset{\circ} R_{h/2} &= \frac{I_{h/2} - I_h}{3}.
\end{align}

Маємо

\begin{equation}
	R_h = - \frac{h^2}{12} \Int_a^b f''(x) \diff x + O(h^4) = O(h^2).
\end{equation}

Отже, якщо застосовувати екстраполяційну формулу Річардсона, то

\begin{equation}
	\label{eq:9.5.4}
	\tilde I_{h/2} = \frac{4}{3} \cdot I_{h/2} - \frac{1}{3} \cdot I_h,
\end{equation}

і $I_h - \tilde I_{h/2} = O(h^4)$. \medskip

Цей принцип застосовується і для формули Сімпсона $m = 4$. Головна частина залишкового члена для цієї формули:

\begin{equation}
	\overset{\circ} R_{h/2} = \frac{I_{h/2} - I_h}{15}.
\end{equation}

\begin{equation}
	\tilde I_{h/2} = \frac{16}{15} \cdot I_{h/2} - \frac{1}{15} \cdot I_h,
\end{equation}

\begin{equation}
	I_h - \tilde I_{h/2} = O(h^6).
\end{equation}

\section{Практична частина}

\subsection{Невласність}

Перш за все зробимо заміну:

\begin{equation}
	\Int_{-1}^1 \frac{\ln(2 + \sqrt[3]{x})}{\sqrt[3]{x}} \diff x = \left| t = \sqrt[3]{x}, x = t^3, \diff x = 3 t^2 \diff t \right| = \Int_{-1}^1 3 t \ln(2 + t) \diff t.
\end{equation}

Таким чином ми звели інтеграл до власного.

\subsection{Квадратурні формули}

Були написані наступні функції для обчислення квадратурних формул:

\inputminted[firstline=6]{python}{../py/integrate.py}

\subsection{Апріорні оцінки похибки}

Були написані наступні функції для обчислення апріорних оцінок похибок:

\inputminted[firstline=4]{python}{../py/apriori_error.py}

Тут нам знадобляться $M_2$ і $M_4$, знайдемо їх:

\begin{equation}
	M_2 = \Max_{-1 \le t \le 1} \left| \frac{\diff^2 f(t)}{\diff t^2} \right|, \quad M_4 = \Max_{-1 \le t \le 1} \left| \frac{\diff^4 f(t)}{\diff t^4} \right|.
\end{equation}

Послідовно знаходимо:

\begin{align}
	\frac{\diff f(t)}{\diff t} &= 3 \left( \frac{t}{t + 2} + \ln (t + 2) \right), \\
	\frac{\diff^2 f(t)}{\diff t^2} &= \frac{3 (t + 4)}{(t + 2)^2}, \\
	\frac{\diff^3 f(t)}{\diff t^3} &= - \frac{3 (t + 6)}{(t + 2)^2}, \\
	\frac{\diff^4 f(t)}{\diff t^4} &= \frac{6 (t + 8)}{(t + 2)^4}, \\
	\frac{\diff^5 f(t)}{\diff t^5} &= - \frac{18 (t + 10)}{(t + 2)^5}.
\end{align}

Як бачимо $\diff^3 f(t) / \diff t^3 < 0$ на $[-1,1]$, тому $M_2$ досягається або в $-1$, або в $1$. Підставляючи знаходимо $f''(-1) = 9$, $f(1) = 5 / 3$, тому $M_2 = 9$. \medskip

Як бачимо $\diff^5 f(t) / \diff t^5 < 0$ на $[-1,1]$, тому $M_4$ досягається або в $-1$, або в $1$. Підставляючи знаходимо $f^{(4)}(-1) = 42$, $f^{(4)}(1) = 2 / 3$, тому $M_4 = 42$.

\subsection{Принцип Рунге}

Були написані наступні функції для обчислення похибки за принципом Рунге:

\inputminted[firstline=7]{python}{../py/runge.py}

\subsection{Формула Річардсона}

Були написані наступні функції для обчислення уточненого значення інтегралу за екстраполяційною формулою Річардсона:

\inputminted[firstline=7]{python}{../py/richardson.py}

\subsection{Програми-драйвери}

Були написані наступні програми-драйвери:

\inputminted[firstline=10, lastline=18]{python}{../py/main.py}
\inputminted[firstline=35, lastline=43]{python}{../py/main.py}
\inputminted[firstline=60, lastline=68]{python}{../py/main.py}
\inputminted[firstline=85]{python}{../py/main.py}

\subsection{Результати}

Тут 

\begin{itemize}
	\item {\tt h} --- кінцевий крок інтегрування;
	\item {\tt I\_true} --- справжнє значення інтегралу;
	\item {\tt I\_h} --- значення інтегралу за відповідною квадратурною формулою для кроку $h$;
	\item {\tt R\_h\_true} --- відхилення {\tt I\_h} від {\tt I\_true};
	\item {\tt I\_half\_h} --- значення інтегралу за відповідною квадратурною формулою для кроку $h / 2$;
	\item {\tt R\_half\_h\_true} --- відхилення {\tt I\_half\_h} від {\tt I\_true};
	\item {\tt R\_runge} --- оцінка відхилення {\tt I\_h} від {\tt I\_true} за принципом Рунге;
	\item {\tt I\_richardson} --- уточнене значення інтегралу за екстраполяційною формулою Річардсона;
	\item {\tt R\_richardson\_true} --- відхилення {\tt I\_richardson} від {\tt I\_true};
	\item {\tt apriori\_error} --- апріорна оцінка відхилення {\tt I\_h} від {\tt I\_true}.
\end{itemize}

\subsubsection{Для формули середніх прямокутників}

\begin{minted}{python}
h                 = 0.00390625
I_true            = 1.0562447009935063
I_h               = 1.0562400624293735
R_h_true          = 4.638564132797285e-06
I_half_h          = 1.0562435413517188
R_half_h_true     = 1.1596417874848441e-06
R_runge           = 1.1596407817708136e-06
I_richardson      = 1.0562447009925007
R_richardson_true = 1.0056400157054668e-12
apriori_error     = 1.1444091796875e-05
\end{minted}

\subsubsection{Для формули трапецій}

\begin{minted}{python}
h                 = 0.00390625
I_true            = 1.0562447009935063
I_h               = 1.0562539781252218
R_h_true          = 9.277131715501596e-06
I_half_h          = 1.0562470202772976
R_half_h_true     = 2.3192837912411335e-06
R_runge           = 2.319282641420154e-06
I_richardson      = 1.0562447009946563
R_richardson_true = 1.1499690089067371e-12
apriori_error     = 2.288818359375e-05
\end{minted}

\subsubsection{Для формули Сімпсона (парабол)}

\begin{minted}{python}
h                 = 0.125
I_true            = 1.0562447009935063
I_h               = 1.0562459003461577
R_h_true          = 1.199352651415353e-06
I_half_h          = 1.056244776246562
R_half_h_true     = 7.525305578681696e-08
R_runge           = 7.49399730419024e-08
I_richardson      = 1.056244701306589
R_richardson_true = 3.130826708996892e-10
apriori_error     = 7.120768229166667e-06
\end{minted}

\end{document}
