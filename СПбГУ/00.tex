\setcounter{section}{-1}
\section{Елементи теорії похибок}

У процесі розв'язування задачі чисельного аналізу спочатку є тільки об'єкт дослідження, потім сптворюється його математична модель, далі вона дискретиизується для комп'ютеризації, і вже тоді програмується відповідний алгоритм. \\

Після програмування зазвичай йде тестування, аналіз результатів тестування і, можливо, уточнення математичної моделі. \\

На всіх описаних етапах можуть виникати похибки.

\subsection{Види похибок}

\begin{enumerate}
	\item \textit{Неусувна} похибка -- похибка, що виникає при математичному моделюванні, і при вимірюванні вхідних даних алгоритму.
	\item Похибка \textit{методу} -- похибка, що виникає через те, що дискретна модель \textit{наближає} математичну.
	\item Похибка \textit{обчислень} -- похибка пов'язана з неточністю збереження чисел з \textit{плаваючою комою} у комп'ютері, а також з неточністю \textit{дійснозначної} комп'ютерної арифметики.
\end{enumerate}

Нехай:
\begin{itemize}
	\item $I$ -- точне значення шуканого об'єкту;

	\item $\tilde I$ -- точний розв'язок математичної моделі;

	\item $\tilde I_h$ -- точкий розв'язок дискретної моделі;

	\item $\tilde I_h^*$ -- розв'язок дискретної моделі з обчислювальними похибками,
\end{itemize}
тоді
\begin{itemize}
	\item $\rho_1 = \tilde I - I$ -- неусувна похибка;

	\item $\rho_2 = \tilde I_h - \tilde I$ -- похибка методу;

	\item $\rho_3 = \tilde I_h^* - \tilde I_h$ -- похибка обчислень;

	\item $\rho = \rho_1 + \rho_2 + \rho_3 = \tilde I_h^* - I$ -- загальна похибка.
\end{itemize}

\begin{example}
	Розглядаємо модель
	\begin{equation}
		\label{eq:0.1}
		A u = f,
	\end{equation}
	де $u \in U$, $f \in F$, $A: U \to F$, $U$, $F$ -- лінійні метричні простори. \\

	Класичним розв'язком є
	\begin{equation}
		\label{eq:0.2}
		u = A^{-1} f.
	\end{equation}

	Тут під \textit{розв'язком} задачі розуміємо
	\begin{equation}
		\label{eq:0.3}
		\tilde u = \arg \inf_{u \in U} \rho_F (A u, f).
	\end{equation}
	Значення $\tilde u$ також називають \textit{узагальненим}, \textit{квазі-} або \textit{псевдо-} розв'язком.
\end{example}

Важиливими характеристиками задачі є \textit{стійкість} та \textit{коректність}. \\

Задача (\ref{eq:0.1}) називається стійкою на парі просторів $(U, F)$ якщо 
\begin{multline}
	\label{eq:0.4}
	\forall \epsilon > 0: \exists \delta(\epsilon) > 0 : \\
	\| A_1 - A_2 \| < \delta \wedge \| f_1 - f_2 \| < \delta \implies \| u_1 - u_2 \| < \epsilon.
\end{multline}

\begin{remark*}
	Тут ми сформулювали твердежння для нормованих просторів, його аналог для метричних має вигляд 
	\begin{multline}
		\label{eq:0.5}
		\forall \epsilon > 0: \exists \delta(\epsilon) > 0 : \\
		\rho_{U\to F} (A_1, A_2) < \delta \wedge \rho_F (f_1, f_2) < \delta \implies \rho_U(u_1, u_2) < \epsilon.
	\end{multline}
\end{remark*}

Задача (\ref{eq:0.1}) називається коректною на парі просторів $(U, F)$ якщо:
\begin{enumerate}
	\item $\forall f \in F: \exists! \tilde u$ (тобто розв'язок (\ref{eq:0.1}));
	\item задача (\ref{eq:0.1}) є стійкою.
\end{enumerate}

\begin{remark*}
	Існування та єдиність розв'язку забезпечують наступні умови:
	\begin{enumerate}
		\item $\exists A^{-1}: F \to U$ (обернене відображення $A$ існує); 
		\item $\| A^{-1} \| < M$ (і ``розтягує'' простір в обмежену кількість разів).
	\end{enumerate}
\end{remark*}