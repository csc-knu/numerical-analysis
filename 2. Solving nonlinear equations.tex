\documentclass[a4paper, 12pt]{article}
\usepackage[utf8]{inputenc}
\usepackage[english, ukrainian]{babel}
\usepackage{amsmath, amssymb}
\usepackage[top = 1 cm, left = 1 cm, right = 1 cm, bottom = 1 cm]{geometry} 

\usepackage{multicol, graphicx}

\usepackage{amsthm}
\theoremstyle{definition}
\newtheorem{problem}{Задача}
\newtheorem{definition}{Визначення}

\allowdisplaybreaks
\setlength\parindent{0pt}
\pagestyle{empty}

\newcommand{\argmax}{\arg\max}
\newcommand{\argmin}{\arg\min}

\newcommand{\dif}{\mathrm{d}}
\newcommand{\dydx}{\dfrac{\dif y}{\dif x}}
\newcommand{\dxdt}{\dfrac{\dif x}{\dif t}}
\newcommand{\dydt}{\dfrac{\dif y}{\dif t}}

\newcommand{\NN}{\mathbb{N}} 
\newcommand{\ZZ}{\mathbb{Z}}
\newcommand{\QQ}{\mathbb{Q}}
\newcommand{\RR}{\mathbb{R}}
\newcommand{\CC}{\mathbb{C}}

\newcommand{\LaReF}[1]{(\ref{#1})}

\renewcommand{\epsilon}{\varepsilon}
\renewcommand{\phi}{\varphi}

\newcommand{\ws}{\text{ }}

\begin{document}
	\setcounter{section}{1}
	
	\section{Методи розв'язання нелінійних рівнянь}
	
	\textit{Постановка задачі}. Нехай маємо рівняння $f(x)=0$, $\bar x$ -- його розв'язок, тобто $f(\bar x)=0$.\\
	
	Задача розв'язання цього рівняння розпадається на етапи:
	\begin{enumerate}
		\item Існування та кількість коренів.
		\item Відділення коренів, тобто розбиття числової вісі на інтервали, де знаходиться один корінь.
		\item Обчислення кореня із заданою точністю $\epsilon$.
	\end{enumerate}

	Для розв'язання перших двох задач використовуються методи математичного аналізу та алгебри, а також графічний метод. Далі розглядаються методи розв'язання третього епату.
	
	\subsection{Метод ділення навпіл}
	Припустимо, що на $[a,b]$ знаходиться лише один корінь рівняння \begin{equation} \label{eq:f(x)=0} f(x)=0 \end{equation} для $f(x)\in C([a,b])$ який необхідно визначити. Нехай $f(a)f(b)<0$. Припустимо, що $f(a)>0$, $f(b)<0$. Покладемо $x_1=\dfrac{a+b}{2}$ і обчислимо $f(x_1)$. Якщо $f(x_1)<0$, то шуканий корінь $\bar x$ знаходиться на інтервалі $(a,x_1)$. Якщо ж $f(x_1)>0$, то $\bar x\in(x_1,b)$. З двох інтервалів $(a,x_1)$ і $(x_1,b)$ вибираємо той, на границях якого $f(x)$ має різні	 знаки, знаходимо точку $x_2$ -- середину вибраного інтервалі, обчислюємо $f(x_2)$, і повторюємо вказаний процес.\\
	
	В результаті отримуємо послідовність інтервалів, що містять шуканий корінь $\bar x$, причому довжина кожного натсупного інтервалу вдвічі менше.\\
	
	Цей процес продовжується доки довжина $b_n-a_n$ отриманого інтервалу $(a_n,b_n)$ не стане меншою за $2\epsilon$. Тоді $x_{n+1}$, як середина інтервалу $(a_n,b_n)$, пов'язана з $\bar x$ нерівністю \begin{equation} \label{eq:xn-xbar} |x_{n+1}-\bar x|<\epsilon. \end{equation} За теоремою Больцано-Коші, ця умова буде виконуватися для деякого $n$. Справді, оскільки \[|b_{k+1}-a_{k+1}=\dfrac12|b_k-a_k|,\] то \begin{equation} \label{eq:xn-xbar 2} |x_{n+1}-\bar x|\le \dfrac1{2^{n+1}}(b-a).\end{equation} Звідси ж отримуємо нерівність для обчислення кількості ітерацій $n$ для виконання умови \LaReF{eq:xn-xbar}: \[n=n(\epsilon)\ge \left[\log\left(\dfrac{b-a}{\epsilon}\right)\right] + 1.\] Степінь збіжності лінійна, тобто геометричної прогресії зі знаменником $q=1/2$.\\
	
	Переваги методу: простота, надійність. Недоліки методу: низька швидкість збіжності, метод не узагальнюється на системи.
	
\end{document}