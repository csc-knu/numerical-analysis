\section{Ітераційні методи для систем}

\subsection{Ітераційні методи розв'язання СЛАР}

Систему
\begin{equation}
	\label{eq:4.1}
	A \vec x = \vec b
\end{equation}
зводимо до вигляду
\begin{equation}
	\label{eq:4.2}
	\vec x = B \vec x + \vec f.
\end{equation}
Будь яка система
\begin{equation}
	\label{eq:4.3}
	\vec x = \vec x - C \cdot (A \vec x - \vec b)
\end{equation}
має вигляд (\ref{eq:4.2}) і при $\det C \ne 0$ еквівалентна системі (\ref{eq:4.1}). Наприклад, для $C = \tau \cdot E$:
\begin{equation}
	\label{eq:4.4}
	\vec x = \vec x - \tau \cdot (A \vec x - \vec b).
\end{equation}


\subsubsection{Метод простої ітерації}

Цей метод застосовується до рівняння (\ref{eq:4.2})
\begin{equation}
	\label{eq:4.5}
	\vec x^{(k+1)} = B \vec x^{(k)} + \vec f,
\end{equation}
де $\vec x^{(0)}$ -- початкове наближення, задано.\\

Ітераційний процес збігається, тобто $\left\| \vec x^{(k)} - \vec x\right\| \xrightarrow[k\to\infty]{} 0$, якщо
\begin{equation}
	\label{eq:4.6}
	\|B\| \le q < 1
\end{equation}
При цьому має місце оцінка
\begin{equation}
	\label{eq:4.7}
	\left\|\vec x^{(n)} - \vec x\right\| \le \dfrac{q^n}{1-q}\cdot\left\|\vec x^{(1)} - \vec x^{(0)}\right\|.
\end{equation}

\subsubsection{Метод Якобі}

Припустимо $\forall i$: $a_{i,i} \ne 0$. Зведемо систему (\ref{eq:4.1}) до вигляду
\[ x_i = -\Sum_{j=1}^{i-1} \dfrac{a_{i,j}}{a_{i,i}} \cdot x_j - \Sum_{j=i+1}^n \dfrac{a_{i,j}}{a_{i,i}} \cdot x_j + \dfrac{b_i}{a_{i,i}}, \quad i=\overline{1,n}. \]

Ітераційний процес запишемо у вигляді
\begin{equation}
	\label{eq:4.8}
	x_i^{(k+1)} = -\Sum_{j=1}^{i-1} \dfrac{a_{i,j}}{a_{i,i}} \cdot x_j^{(k)} - \Sum_{j=i+1}^n \dfrac{a_{i,j}}{a_{i,i}} \cdot x_j^{(k)} + \dfrac{b_i}{a_{i,i}}, \quad k = 0,1,\ldots, \quad i=\overline{1,n}.
\end{equation}

Ітераційний процес збігається до розв’язку, якщо виконується умова
\[ \forall i: \Sum_{\substack{j = 1 \\ i \ne j}}^n |a_{i,j}| \le |a_{i,i}|. \]

Це умова діагональної переваги матриці $A$. Якщо ж
\begin{equation}
	\label{eq:4.9}
	\forall i: \Sum_{\substack{j = 1 \\ i \ne j}}^n |a_{i,j}| \le q\cdot|a_{i,i}|, \quad 0 \le q < 1.
\end{equation}
то має місце оцінка точності:
\[ \|\vec x^{(n)} - \vec x\| \le \dfrac{q^n}{1-q}\cdot\|\vec x^{(0)}-\vec x\|. \]

\subsubsection{Метод Зейделя}
В компонентному вигляді ітераційний метод Зейделя записується так:
\begin{equation}
	\label{eq:4.10}
	x_i^{(k+1)} = -\Sum_{j=1}^{i-1} \dfrac{a_{i,j}}{a_{i,i}} \cdot x_j^{(k+1)} - \Sum_{j=i+1}^n \dfrac{a_{i,j}}{a_{i,i}} \cdot x_j^{(k)} + \dfrac{b_i}{a_{i,i}}, \quad k = 0,1,\ldots, \quad i=\overline{1,n}.
\end{equation}

На відміну від методу Якобі на $k$-му-кроці попередні компоненти розв'язку беруться з $k+1$-ої ітерації. \\

Достатня умова збіжності методу Зейделя -- $A^T = A > 0$.

\subsubsection{Матрична інтерпретація методів Якобі і Зейделя}

Подамо матрицю $A$ у вигляді \[ A = A_1 + D + A_2, \]
де $A_1$ -- нижній трикутник матриці $A$, $A_2$ -- верхній трикутник матриці $A$, $D$ -- її
діагональ. Тоді систему (\ref{eq:4.1}) запишемо у вигляді \[ D \vec x = A_1 \vec x + A_2 \vec x + \vec b,\]
або
\[ \vec x = D^{-1} A_1 \vec x + D^{-1} A_2 \vec x + D^{-1} \vec b,\]
 
Матричний запис методу Якобі:
\[ \vec x^{(k+1)} = D^{-1} A_1 \vec x^{(k)} + D^{-1} A_2 \vec x^{(k)} + D^{-1} \vec b,\]
методу Зейделя:
\[ \vec x^{(k+1)} = D^{-1} A_1 \vec x^{(k+1)} + D^{-1} A_2 \vec x^{(k)} + D^{-1} \vec b,\]

Необхідна і достатня умова збіжності методу Якобі: всі корені рівняння \[\det(D + \lambda(A_1 + A_2 )) = 0\] по модулю більше 1. \\

Необхідна і достатня умова збіжності методу Зейделя: всі корені рівняння \[\det(A_1 + D + \lambda A_2) = 0\] по модулю більше 1.

\subsubsection{Однокрокові (двошарові) ітераційні методи}

Канонічною формою однокрокового ітераційного методу розв'язку СЛАР є його запис у вигляді
\begin{equation}
	\label{eq:4.11}
	B_k \dfrac{\vec x^{(k+1)} - \vec x^{(k)}}{\tau_{k+1}} + A \vec x^{(k)} = \vec b,
\end{equation}

Тут $\{B_k\}$ -- послідовність матриць (пере-обумовлюючі матриці), що задають ітераційний метод на кожному кроці; $\{\tau_{k+1}\}$ -- ітераційні параметри. \\

Якщо $B_k = E$, то ітераційний процес називається \textit{явним}
\[ \vec x^{(k+1)} = \vec x^{(k)} - \tau_{k+1} \left(A \vec x^{(k)} + \vec b\right). \]
Якщо $B_k \ne E$, то ітераційний процес називається \textit{неявним}
\[ B_k \vec x^{(k+1)} = F^k. \]

У цьому випадку на кожній ітерації необхідно розв'язувати СЛАР. \\

Якщо $\tau_{k+1} \equiv \tau$, $B_k \equiv B$, то ітераційний процес називається \textit{стаціонарним}; інакше -- \textit{нестаціонарним}. \\

Методам, що розглянуті вище відповідають:
\begin{itemize}
	\item методу простої ітерації: $B_k = E$, $\tau_{k+1} = \tau$;
	\item методу Якобі: $B_k = D$, $\tau_{k+1} = 1$.;
	\item методу Зейделя: $B_k = D + A_1$, $\tau_{k+1} = 1$.
\end{itemize}

\subsubsection{Збіжності стаціонарних ітераційних процесів у випадку симетричних матриць}

Розглянемо випадок симетричних матриць $A^T=A$ і стаціонарний ітераційний процес $B_k \equiv E$, $\tau_{k+1} \equiv \tau$. \\

Нехай для $A$ справедливі нерівності
\begin{equation}
	\label{eq:4.12}
	\gamma_1 E \le A \le \gamma_2 E, \quad \gamma_1, \gamma_2 > 0.
\end{equation}

Тоді при виборі $\tau = \tau_0 = \frac{2}{\gamma_1 + \gamma_2}$ ітераційний процес збігається. Найбільш точним значенням $\gamma_1$, $\gamma_2$ при яких виконуються обмеження (\ref{eq:4.12}) є $\gamma_1 = \min \lambda_i(A)$, $\gamma_2 = \max \lambda_i(A)$. Тоді 
\[q = q_0 = \dfrac{\gamma_2 - \gamma_1}{\gamma_2 + \gamma_1} = \dfrac{1-\xi}{1+\xi}, \quad \xi = \dfrac{\gamma_1}{\gamma_2}.\]
(Зауважимо, що аналогічно обчислюється $q$ і для методу релаксації розв'язання нелінійних рівнянь, де $\gamma_1 = m = \min |f'(x)|$, $\gamma_2 = M_1 = \max|f'(x)|$) і справедлива оцінка\[ \|\vec x^{(n)} - \vec x\| \le \dfrac{q^n}{1-q} \cdot \|\vec x^{(0)} - \vec x\|. \]

Явний метод з багатьма параметрами $\{\tau_k\}$:
\[ B \equiv E, \quad \{\tau_k\}: \Min_\tau q(\tau), \quad n=n(\epsilon)\to\min,\]
які обчислюються за допомогою нулів багаточлена Чебишова, називаються ітераційним методом з чебишевським набором параметрів.

\subsubsection{Метод верхньої релаксації}

Узагальненням методу Зейделя є метод верхньої релаксації: \[ (D + \omega A_1) \cdot\dfrac{\vec x^{(k+1)} + \vec x^{(k)}}{\omega} + A \vec x^{(k)} = \vec b,\]
де $D$ -- діагональна матриця з елементами $a_{i,i}$ по діагоналі. $\omega > 0$ -- заданий числовий параметр. \\

Тепер $B = D + \omega A_1$, $\tau = \omega$. Якщо $A^T = A > 0$, то метод верхньої релаксації збігається при умові $0 < \omega < 2$. Параметр підбирається експериментально з умови мінімальної кількості ітерацій. 

\subsubsection{Методи варіаційного типу}

До цих методів відносяться: метод мінімальних нев’язок, метод мінімальних поправок, метод найшвидшого спуску, метод спряжених градієнтів. Вони дозволяють обчислювати наближення без використання апріорної інформації про $\gamma_1$, $\gamma_2$ в (\ref{eq:4.12}). \\

Нехай $B = E$. Для методу мінімальних нев’язок параметри $\tau_{k+1}$ обчислюються з умови 
\[ \left\|\vec r^{(k+1)}\right\|^2 = \left\|\vec r^{(k)}\right\|^2 - 2\tau_{k+1}\cdot\left(\vec r^{(k)}, A\vec r^{(k)}\right) + \tau_{k+1}^2 \cdot\left\|A\vec r^{(k)}\right|^2 \to \min. \]

Тому \[ \tau_{k+1} = \dfrac{\left(A\vec r^{(k)}, \vec r^{(k)}\right) }{\left\|\vec r^{(k)}\right\|^2},\] 
де $\vec r^{(k)} = A \vec x^{(k)} - \vec b$ -- нев'язка. \\

Умова для завершення ітераційного процесу: \[ \left\|\vec r^{(n)}\right\| < \epsilon.\]

Швидкість збіжності цього методу співпадає із швидкістю методу простої ітерації з одним оптимальним параметром $\tau_0 = \frac{2}{\gamma_1+\gamma_2}$. \\

Аналогічно будуються методи з $B \ne E$. Матриця $B$ називається переобумовлювачем і дозволяє підвищити швидкість збіжності ітераційного процесу. Його вибирають з умов 
\begin{enumerate}
	\item легко розв’язувати СЛАР $B \vec x^{(k)} = F_k$ (діагональний, трикутній, добуток трикутніх та інше); 
	\item зменшення числа обумовленості матриці $B^{-1}A$ у порівнянні з $A$.
\end{enumerate}

\subsection{Методи розв’язання нелінійних систем}

Розглянемо систему рівнянь
\[ \left\{ \begin{aligned} & f_1(x_1, \ldots, x_n) = 0, \\ & \ldots \\ & f_n(x_1,\ldots,x_n) = 0. \end{aligned} \right. \]

Перепишемо її у векторному вигляді: 
\begin{equation}
	\label{eq:4.13}
	\vec f(\vec x) = 0.
\end{equation}

\subsubsection{Метод простої ітерації}

В цьому методі рівняння (\ref{eq:4.13}) зводиться до еквівалентного вигляду
\begin{equation}
	\label{eq:4.14}
	\vec x = \vec \Phi(\vec x).
\end{equation}

Ітераційний процес представимо у вигляді:
\begin{equation}
	\label{eq:4.15}
	\vec x^{(k+1)} = \vec \Phi\left(\vec x^{(k)}\right).
\end{equation}
початкове наближення $\vec x^{(0)}$ -- задано. \\

Нехай оператор $\vec \Phi$ визначений на множині $H$. За теоремою про стискуючі відображення ітераційний процес (\ref{eq:4.15}) сходиться, якщо виконується умова
\begin{equation}
	\label{eq:4.16}
	\left\| \vec \Phi(\vec x) - \vec \Phi(\vec y) \right\| \le q \cdot \|\vec x - \vec y\|, \quad 0 < q < 1, 
\end{equation}
або
\begin{equation}
	\label{eq:4.17}
	\left\| \vec \Phi'(\vec x)\right\| \le q < 1, 
\end{equation}
де $\vec x\in U_r$, $\vec \Phi'(\vec x) = \left(\frac{\partial \phi_i}{\partial x_j}\right)_{i,j=1}^n$. Для похибки справедлива оцінка
\[ \left\| \vec x^{(m)} - \vec x\right\| \le \dfrac{q^n}{1 - q} \cdot\left\|\vec x^{(0)} - \vec x\right\|.\]

Частинним випадком методу простої ітерації є метод релаксації для рівняння (\ref{eq:4.13}):
\[ \vec x^{(k+1)} = \vec x^{(k)} - \tau \cdot \vec F\left(\vec x^{(k)}\right), \]
де $\tau < \frac{2}{\left\|\vec F'(\vec x)\right\|}$.

\subsubsection{Метод Ньютона}

Розглянемо рівняння
\[ \vec F(\vec x) = 0. \]

Представимо його у вигляді
\begin{equation}
	\label{eq:4.18}
	\vec F\left(\vec x^{(k)}\right) + \vec F'\left(\vec \xi^{(k)}\right)\cdot\left(\vec x - \vec x^{(k)}\right) = 0,
\end{equation}
де $\vec \xi^{(k)} = \vec x^{(k)} + \theta_k \cdot \left(\vec x^{(k)} - \vec x\right)$, $0 < \theta_k < 1$. Тут  $\vec F'(\vec x) = \left(\frac{\partial f_i}{\partial x_j}\right)_{i,j=1}^n$ -- матриця Якобі для $\vec F(\vec x)$. Можемо наближено вважати $\vec \xi^{(k)} \approx \vec x^{(k)}$. Тоді з (\ref{eq:4.18}) матимемо
\begin{equation}
	\label{eq:4.19}
	\vec F\left(\vec x^{(k)}\right) + \vec F'\left(\vec x^{(k)}\right) \cdot\left(\vec x^{(k+1)} - \vec x^{(k)}\right) = 0.
\end{equation}
Ітераційний процес представимо у вигляді:
\begin{equation}
	\label{eq:4.20}
	\vec x^{(k+1)} = \vec x^{(k)} - \vec F'\left(\vec x^{(k)}\right)^{-1} \cdot \vec F\left(\vec x^{(k)}\right). 
\end{equation}

Для реалізації методу Ньютона потрібно, щоб існувала обернена матриця \[\vec F'\left(\vec x^{(k)}\right)^{-1}. \]

Можна не шукати обернену матрицю, а розв’язувати на кожній ітерації СЛАР
\begin{equation}
	\label{eq:4.21}
	\left\{
		\begin{aligned}
			& A_k \vec z^{(k)} = \vec F\left(\vec x^{(k)}\right), \\
			& \vec x^{(k + 1)} = \vec x^{(k)} - \vec z^{(k)},
		\end{aligned}
		\quad k=0,1,2,\ldots
	\right.
\end{equation}
де $\vec x^{(0)}$ -- задано, а матриця $A_k = \vec F'\left(\vec x^{(k)}\right)$. \\

Метод має квадратичну збіжність, якщо добре вибрано початкове наближення. Складність методу (при умові використання методу Гаусса розв'язання СЛАР (\ref{eq:4.21}) на кожній ітерації $Q_n = \frac23 n^3+O(n^2)$, де $n$ -- розмірність системи (\ref{eq:4.13}).

\subsubsection{Модифікований метод Ньютона}

Ітераційний процес має вигляд :
\[ \vec x^{(k+1)} = \vec x^{(k)} - \vec F'\left(\vec x^{(0)}\right)^{-1} \cdot\vec F\left(\vec x^{(k)}\right). \]

Тепер обернена матриця обчислюється тільки на нульовій ітерації. На інших -- обчислення нового наближення зводиться до множення матриці $A_0 = \vec F'\left(\vec x^{(0)}\right)^{-1}$ на вектор $\vec F\left(\vec x^{(k)}\right)$ та додавання до $\vec x^{(k)}$. \\

Запишемо метод у вигляді системи лінійних рівнянь (аналог (\ref{eq:4.21}))
\begin{equation}
	\label{eq:4.22}
	\left\{
		\begin{aligned}
			& A_0 \vec z^{(k)} = \vec F\left(\vec x^{(k)}\right), \\
			& \vec x^{(k + 1)} = \vec x^{(k)} - \vec z^{(k)},
		\end{aligned}
		\quad k=0,1,2,\ldots
	\right.
\end{equation}
Оскільки матиця $A_0$ розкладається на трикутні (або обертається) один раз, то складність цього методу на одній ітерації (окрім нульової) $Q_n = O(n^2)$. Але цей метод має лінійну швидкість збіжності. \\

Можливе циклічне застосування модифікованого методу Ньютона, тобто коли обернену матрицю похідних шукаємо та обертаємо через певне число кроків ітераційного процесу. \\

\begin{problem}
	Побудувати аналог методу січних для систем нелінійних рівнянь.
\end{problem}